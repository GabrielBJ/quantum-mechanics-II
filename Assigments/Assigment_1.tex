\documentclass[12 pt]{article}

% Packages for mathematical symbols and equations
\usepackage{amsmath}
\usepackage{amssymb}
\usepackage{listings}
\usepackage{color}
\usepackage{setspace}
\usepackage{acro}
\usepackage{amsmath}
\usepackage{amsthm}
\usepackage{enumerate}
\usepackage{graphicx}
\usepackage{subcaption}
\usepackage{physics}
\usepackage{cancel}
\usepackage{empheq}
\usepackage{tikz}
\usepackage{color}
\usepackage[english]{babel}
\usepackage{bbold}
\usepackage{amssymb}
\usepackage{titlesec}
\usepackage{adjustbox}
\usepackage{float}
\numberwithin{equation}{section}
\usepackage{bm}
\usepackage{wrapfig}
\usepackage{relsize}
\usepackage[letterpaper ,top=1in,bottom=1in,left=1in,right=1in,marginparwidth=1.75cm]{geometry}
\usepackage[colorlinks=true, allcolors=blue]{hyperref}
\usetikzlibrary{calc,trees,positioning,arrows,chains,shapes.geometric,%
    decorations.pathreplacing,decorations.pathmorphing,shapes,%
    matrix,shapes.symbols}
\include{acronyms}
%\geometry{top=1.3in,bottom=1.3in}
\hypersetup{
    colorlinks,
    citecolor=black,
    filecolor=black,
    linkcolor=black,
  urlcolor=black}



% Set page margins

\title{Quantum Mechanics II\\
Assignment I}
\author{Gabriel Balarezo}
\date{\today}

\begin{document}

\maketitle

\section*{Problem 6.27}

Suppose a Hamiltonian $H$, for a  particular quantum system is a function of some parameter
$\lambda$; let $E_n(\lambda)$ and $\psi_n(\lambda)$ be the eigenvalues and eigenfunctions of $H
(\lambda)$. The \textbf{Feynman-Hellmann theorem} states that
\[
  \frac{\partial E_n}{\partial \lambda} = \bigg<\psi_n\bigg|\frac{\partial H}{\partial \lambda}
  \bigg|\psi_n\bigg>
\]
(assuming either that $E_n$ is nondegenerate, or if degenerate that $\psi_n$'s are the 
"good" linear combinations of the degenerate eigenfunctions).
\begin{enumerate}[a)]
  \item Prove the Feynman-Hellmann theorem. Hint: Use Equation 6.9.
  \item Apply it to the one-dimensional harmonic oscillator, (i) using $\lambda=\omega$
    (this yields a formula for the expectation value of $V$ ), (ii) using $\lambda=\hbar$ 
    (this yields $\big<T\big>)$, and (iii) using $\lambda=m$ (this yields a relation between 
    $\langle T\rangle$ and $\big<V\big>)$. Compare your answers to Problem 2.37 and the virial
    theorem predictions (Problem 3.53).
\end{enumerate}
\subsection*{Solution}
\textbf{Part a}\\
\\
Let's suppose $H_\lambda$ a hamiltonian which depends on a parameter $\lambda$. Then let $E_\lambda$
be the eigenenergies, and $\psi_\lambda$ be the normalised
eigenstates which also depend on $\lambda$.\\
\\
Then we can write the following 
\[
  H_\lambda \big|\psi_\lambda\big> = E_\lambda\big|\psi_\lambda\big>
\]
and using Equation 6.9 as suggested,we can write the following equation
\[
  E_\lambda = \big<\psi_\lambda\big|H_\lambda\big|\psi_\lambda\big>
\]
Now let's differentiate w.r.t $\lambda$ in both sides
\begin{align*}
  \frac{\partial}{\partial \lambda} E_\lambda = \frac{\partial}{\partial \lambda}
  \big<\psi_\lambda\big|H_\lambda\big|\psi_\lambda\big>
\end{align*}
using chain rule
\begin{align*}
  \frac{\partial E_\lambda}{\partial \lambda} =
  \bigg<\frac{\partial \psi_\lambda}{\partial \lambda}\bigg|H_\lambda\bigg|\psi_\lambda\bigg>
  + \bigg<\psi_\lambda\bigg|\frac{\partial H_\lambda}{\partial \lambda}\bigg|\psi_\lambda\bigg>
  +  \bigg<\psi_\lambda\bigg|H_\lambda\bigg|\frac{\partial \psi_\lambda}{\partial \lambda}\bigg>
\end{align*}
we know that
\[
  H_\lambda\big|\psi_\lambda\big> =  E_\lambda\big|\psi_\lambda\big>,\quad \text{and}
  \quad\big<\psi_\lambda\big|H_\lambda = E_\lambda\big<\psi_\lambda\big| 
\]
then 
\begin{align*}
  \frac{\partial E_\lambda}{\partial \lambda}&=
  E_\lambda\bigg<\frac{\partial \psi_\lambda}{\partial \lambda}\bigg|\psi_\lambda\bigg>
  + \bigg<\psi_\lambda\bigg|\frac{\partial H_\lambda}{\partial \lambda}\bigg|\psi_\lambda\bigg>
  + E_\lambda\bigg<\psi_\lambda\bigg|\frac{\partial \psi_\lambda}{\partial \lambda}\bigg>\\
  &=
   \bigg<\psi_\lambda\bigg|\frac{\partial H_\lambda}{\partial \lambda}\bigg|\psi_\lambda\bigg>
   + E_\lambda\left(\bigg<\frac{\partial \psi_\lambda}{\partial \lambda}\bigg|\psi_\lambda\bigg>
   + \bigg<\psi_\lambda\bigg|\frac{\partial \psi_\lambda}{\partial \lambda}\bigg>\right)\\
  &= \bigg<\psi_\lambda\bigg|\frac{\partial H_\lambda}{\partial \lambda}\bigg|\psi_\lambda\bigg>
  + E_\lambda\left(\frac{\partial }{\partial \lambda}\bigg<\psi_\lambda\bigg|\psi_\lambda\bigg>
  +  \frac{\partial }{\partial \lambda}\bigg<\psi_\lambda\bigg|\psi_\lambda\bigg>\right), 
  \quad \big<\psi_\lambda\big|\psi_\lambda\big> = 1\\
  &= \bigg<\psi_\lambda\bigg|\frac{\partial H_\lambda}{\partial \lambda}\bigg|\psi_\lambda\bigg>
  + E_\lambda\cancelto{0}{\left(\frac{\partial }{\partial \lambda} (1)+\frac{\partial }{\partial \lambda} (1)
  \right)}
\end{align*}
Finally we have that: 
\[
 \frac{\partial E_\lambda}{\partial \lambda}=\bigg<\psi_\lambda\bigg|\frac{\partial
 H_\lambda}{\partial \lambda}\bigg|\psi_\lambda\bigg> 
\]
or 
\[
  \frac{\partial E_n}{\partial \lambda}=\bigg<\psi_n\bigg|\frac{\partial
 H_\lambda}{\partial \lambda}\bigg|\psi_n\bigg>  
\]
as stated in the problem. \\
\\
\textbf{Part b}\\
\\
For the 1D Harmonic oscillator:
\[
  H = -\frac{\hbar^2}{2m}\frac{d^2}{dx^2} +\frac{1}{2}m\omega^2 x^2,\quad 
  E_n = \hbar\omega\left(n+\frac{1}{2} \right), \quad n=0,1,2,3,\cdots
\]
\\
\textbf{ i)}  $\mathbf{\lambda = \omega}$\\
 \\
 Compute the derivatives:
 \begin{align*}
   \frac{\partial E_n}{\partial \omega} &= \hbar\left(n+\frac{1}{2}\right)\\
  \frac{\partial H}{\partial \omega}&= m\omega x^2
\end{align*}
Aplying the Feynman-Hellmann theorem: 
\[
  \hbar\left(n+\frac{1}{2}\right) = \bigg<m\omega x^2\bigg>
\]
we can notice that the right hand side can be written as 
\[
  m\omega x^2 = \frac{2}{\omega} V
\]
then 
\[
  \hbar\left(n+\frac{1}{2}\right) = \frac{2}{\omega}\bigg<V\bigg>
\]
Finally we get
\[
  \boxed{ \frac{\hbar\omega}{2}\left(n+\frac{1}{2}\right) = \bigg<V\bigg>}
\]
which is the expectation value of the potential energy
for a 1D harmonic oscillator for any value of n.\\
\\
\textbf{ii) }$\lambda = \hbar$\\
\\
Compute the derivatives:
\begin{align*}
  \frac{\partial E_n}{\partial \hbar} &= \omega\left(n+\frac{1}{2}\right)\\
  \frac{\partial H}{\partial \omega}&= \frac{-\hbar}{m}\frac{d}{dx}
\end{align*}
Aplying the Feynman-Hellmann theorem:
\[
 \omega\left(n+\frac{1}{2}\right) = \bigg< \frac{-\hbar}{m}\frac{d^2}{dx^2}
\bigg> 
\]
we can rewrite the right hand side as follows
\[
  \frac{-\hbar}{m}\frac{d^2}{dx^2} = \frac{2}{\hbar} T
\]
then 
\[
 \omega\left(n+\frac{1}{2}\right) = \frac{2}{\hbar}\bigg<T\bigg> 
\]
from here we get
\[
  \boxed{\frac{\hbar\omega}{2}\left(n+\frac{1}{2}\right) = \bigg<T\bigg>}
\]
which is the expectation value of the kinetic energy for a 1D harmonic oscillator. \\
\\
\\
\textbf{iii)} $\lambda= m$\\
\\
Get the derivatives:
\begin{align*}
  \frac{\partial E_n}{\partial m} &= 0\\
  \frac{\partial H}{\partial m} &= \frac{\hbar^2}{2m^2}\frac{d^2}{dx^2} + \frac{1}{2}\omega^2x^2
\end{align*}
Apply the Feynman-Hellmann theorem
\[
  0 = \bigg<\frac{\hbar^2}{2m^2}\frac{d^2}{dx^2}\bigg> +
  \bigg<\frac{1}{2}\omega^2x^2 \bigg>
\]
the expresions in the right hand side can be written as 
\begin{align*}
  \frac{\hbar^2}{2m^2}\frac{d^2}{dx^2} &= -\frac{1}{m}T\\
  \frac{1}{2}\omega^2x^2 &= \frac{1}{m}V
\end{align*}
then, the theorem becomes 
\[
 0 = -\frac{1}{m}\bigg<T\bigg> + \frac{1}{m}\bigg<V\bigg>
\]
from here, we can derive the following expresion: 
\[
  \boxed{ \bigg<T\bigg> = \bigg<V\bigg>}
\]
which is the relation proposed by the virial theorem.\\
\\
Now, let's compare our results with those from Problem 2.37, and the virial theorem predictions 
in Problem 3.53. \\
\\
The results from \textbf{(i)} and \textbf{(ii)} agree with those we obtained using the Feynman-
Hellmann theorem, which seems to be a more efficient way to compute those expectation values. 
\\
In addition, the result from part \textbf{(iii)} effectively agree with the virial theorem 
predictions. 
%%%%%%%%%%%%%%%%%%%%%%%%%%%%%%%%%%%%%%%%%%%%%%%%%%%%%%%%%%%%%%%%%%%%%%%%%%%%%%%%%%%%%%%%%%%%%%%
%%%%%%%%%%%%%%%%%%%%%%%%%%%%%%%%%%%%%%%%%%%%%%%%%%%%%%%%%%%%%%%%%%%%%%%%%%%%%%%%%%%%%%%%%%%%%%%
%%%%%%%%%%%%%%%%%%%%%%%%%%%%%%%%%%%%%%%%%%%%%%%%%%%%%%%%%%%%%%%%%%%%%%%%%%%%%%%%%%%%%%%%%%%%%%%
\section*{Problem 6.28}
The Feynman-Hellmann theorem (Problem 6.27) can be used to determine the expectation values of 
$1 / r$ and $1 / r^2$ for hydrogen.  The effective Hamiltonian for the radial wave functions
is (Equation 4.53)
\[
H=-\frac{\hbar^2}{2 m} \frac{d^2}{d r^2}+\frac{\hbar^2}{2 m} \frac{l(l+1)}{r^2}-
\frac{e^2}{4 \pi \epsilon_0} \frac{1}{r},
\]
and the eigenvalues (expressed in terms of l) are (Equation 4.70)
\[
E_n=-\frac{m e^4}{32 \pi^2 \epsilon_0^2 \hbar^2\left(j_{\max }+l+1\right)^2}
\]
\begin{enumerate}[a)]
  \item Use $\lambda=e$ in the Feynman-Hellmann theorem to obtain $\langle 1 / r\rangle$. 
    Check your result against Equation 6.54 .
  \item  Use $\lambda=l$ to obtain $\left\langle 1 / r^2\right\rangle$. Check your answer 
    with Equation 6.55.
\end{enumerate}
\subsection*{Solution}
\textbf{Part a}\\
\\
First let's compute the necessary derivatives setting $\lambda = e$:
\begin{align*}
  \frac{\partial E_n}{\partial e} &= -\frac{1}{8}\frac{me^3}{\pi^2\epsilon_0^2\hbar^2
  (j_{max}+l+1)^2} \\
  \frac{\partial H}{\partial e} &= -\frac{e}{2\pi\epsilon_0}\frac{1}{r}
\end{align*}
then, using the Feynman-Hellmann theorem we have 
\begin{align*}
  -\frac{1}{8}\frac{me^3}{\pi^2\epsilon_0^2\hbar^2
  (j_{max}+l+1)^2} &= \bigg< -\frac{e}{2\pi\epsilon_0}\frac{1}{r}\bigg>\\
                   &= -\frac{e}{2\pi\epsilon_0}\bigg<\frac{1}{r}\bigg>
\end{align*}
Solving for $\bigg<\frac{1}{r}\bigg>$
\[
  \bigg<\frac{1}{r}\bigg> = \frac{1}{4}\frac{me^3}{\pi\epsilon_0\hbar^2(j_{max}+l+1)^2}
\]
we can rewrite this expresion using the Bohr radius:
\[
  a_0 = \frac{4\pi\epsilon_0\hbar^2}{me^2}
\]
then, 
\[
  \bigg<\frac{1}{r}\bigg> = \frac{1}{a_0(j_{max}+l+1)^2}
\]
let $n = j_{max}+l+1$
\[
  \boxed{\bigg<\frac{1}{r}\bigg> = \frac{1}{a_0 n^2}},\qquad \parbox{0.4\textwidth}{
    this result agrees with that stated in Equation 6.54.
  }
\]
\textbf{Part b}\\
\\
Now we set $\lambda = l$, let's get the derivatives: 
\begin{align*}
  \frac{\partial E_n}{\partial l} &= \frac{2me^4}{32\pi^2\epsilon_0^2\hbar^2(j_{max}+l+1)^3}\\
  \frac{\partial H}{\partial l}&= \frac{\hbar^2}{2m} \frac{2l+1}{r^2}
\end{align*}
Now, using the Feynman-Hellmann theorem we get 
\begin{align*}
 \frac{2me^4}{32\pi^2\epsilon_0^2\hbar^2(j_{max}+l+1)^3}
 &= \bigg<\frac{\hbar^2}{2m} \frac{2l+1}{r^2} \bigg>\\
 &=\frac{\hbar^2(2l+1)}{2m}\bigg<\frac{1}{r^2} \bigg>
\end{align*}
solving for $\bigg<\frac{1}{r^2}\bigg>$
\begin{align*}
  \bigg<\frac{1}{r^2}\bigg>&= \frac{2me^4}{32\pi^2\epsilon_0^2\hbar^2(j_{max}+l+1)^3}
  \frac{2m}{\hbar^2(2l+1)}\\
                           &= \frac{m^2e^4}{16\pi^2\epsilon_0^2\hbar^4
                           (j_{max}+l+1)^3}\frac{1}{(l+1/2)}
\end{align*}
using Bohr radius and using $n = j_{max}+l+1$ we have
\[
  \boxed{ \bigg<\frac{1}{r^2}\bigg> =\frac{1}{(l+1/2)n^3a_0^2} } 
  \qquad \parbox{0.3\textwidth}{This result agrees with that stated in Equation 6.55.}
\]
%%%%%%%%%%%%%%%%%%%%%%%%%%%%%%%%%%%%%%%%%%%%%%%%%%%%%%%%%%%%%%%%%%%%%%%%%%%%%%%%%%%%%%%%%%%%%%%%%
%%%%%%%%%%%%%%%%%%%%%%%%%%%%%%%%%%%%%%%%%%%%%%%%%%%%%%%%%%%%%%%%%%%%%%%%%%%%%%%%%%%%%%%%%%%%%%%%%
%%%%%%%%%%%%%%%%%%%%%%%%%%%%%%%%%%%%%%%%%%%%%%%%%%%%%%%%%%%%%%%%%%%%%%%%%%%%%%%%%%%%%%%%%%%%%%%%%
\section*{Problem 6.30}
\begin{enumerate}[a)]
  \item Plug $s=0, s=1, s=2$, and $s=3$ into Kramers' relation (Equation 6.96) to obtain 
    formulas for $\left\langle r^{-1}\right\rangle,\langle r\rangle,\left\langle 
    r^2\right\rangle$, and $\left\langle r^3\right)$. Note that you could continue 
    indefinilely, to find any positive power.
  \item  In the other direction, however, you hit a snag. Put in $s=-1$, and show that all 
    you get is a relation between $\left\langle r^{-2}\right\rangle$ and $\left\langle
    r^{-3}\right\rangle$.
  \item But if you can get $\big<r^{-2}\big>$ by some other means, you can apply the Kramers' 
    relation to obtain the rest of the negative powers. Use Equation 6.55 (which is derived in 
    Problem 6.28) to determine $\big<r^{-3}\big>$, and check your answer against Equation 6.63.
\end{enumerate}
Kramers' relation 
\[
  \frac{s+1}{n^2}\big<r^s\big> - (2s+1)a_0 \big<r^{s-1}\big> + \frac{s}{4}\left[
  (2l+1)^2-s^2\right]a_0^2\big<r^{s-2}\big> = 0
\]
\textbf{Part a}\\
\\
\textbf{Pluggin $s=0$}\\
\\
With $s = 0$, Kramers' relation becomes 
\[
  \frac{1}{n^2} - a_0\big<r^{-1}\big> = 0
\]
which yields to:
\[
  \boxed{\bigg<\frac{1}{r}\bigg> = \frac{1}{n^2 a_0}}
\]
\textbf{Pluggin $s= 1$}\\
\\
with $s=1$, Kramers' relation becomes: 
\[
  \frac{2}{n^2}\big<r\big> - 3a_0 + \frac{1}{4}\left[(2l+1)^2-1\right]a_0^2\big<r^{-1}\big>=0 
\]
solve for $\big<r\big>$, and use result obtained for $\bigg<\frac{1}{r}\bigg>$
\begin{align*}
  \big<r\big> &= \frac{n^2}{2}\left[3a_0- \frac{1}{4}\left[(2l+1)^2-1\right]a_0^2\big<r^{-1}
  \big> \right]\\
  &= \frac{n^2}{2}\left[3a_0- \frac{1}{4}\left[(2l+1)^2-1\right]a_0^2
  \left(\frac{1}{a_0n^2}\right) \right]\\
  &= \frac{n^2}{2}\left[3a_0- \frac{1}{4n^2}\left[(2l+1)^2-1\right]a_0\right]\\
    \Aboxed{\big<r\big> &= \frac{a_0}{2}\left[3n^2- l(l+1)\right]}
  \end{align*}
\textbf{Pluggin $s=2$}
\\
with $s=2$, Kramers' relation becomes:
\[
  \frac{3}{n^2}\big<r^2\big>-5a_0\big<r\big> + \frac{1}{2}\left[4l(l+1)-3\right]a_0^2 = 0
\]
solve for $\big<r^2\big>$ and use the result obtained for $\big<r\big>$
\begin{align*}
  \big<r^2\big>&= \frac{n^2}{3}\left[5a_0\big<r\big> - \frac{1}{2}\left[4l(l+1)-3\right]a_0^2 
  \right]\\
&=\frac{n^2}{3}\left[5a_0\left(\frac{a_0}{2}\left[3n^2- l(l+1)\right]\right)
- \frac{1}{2}\left[4l(l+1)-3\right]a_0^2\right]\\
&= \frac{n^2a_0^2}{6}\left[15n^2 -5l(l+1) - 4l(l+1)+3\right]\\
    \Aboxed{\big<r^2\big>&=\frac{n^2a_0^2}{2}\left[5n^2 -3l(l+1)+1\right] }
\end{align*}
\textbf{Pluggin $s=3$}\\
\\
with $s=3$, Kramers' relation becomes:
\[
  \frac{4}{n^2}\big<r^3\big> - 7a_0\big<r^2\big> + \frac{3}{4}\left[4l(l+1)-8\right]a_0^2\big<r
  \big>=0
\]
solve for $\big<r^3\big>$. and use the results obtained for $\big<r\big>$ and $\big<r^2\big>$
\begin{align*}
  \big<r^3\big>&=\frac{n^2}{4}\left[7a_0\big<r^2\big>-\frac{3}{4}\left[4l(l+1)-8\right]
  a_0^2\big<r\big>\right]\\
  &=\frac{n^2}{4}\left[7a_0\left[\frac{n^2a_0^2}{2}\left(5n^2 -3l(l+1)+1\right) \right] 
  -\frac{3}{4}\left(4l(l+1)-8\right)a_0^2\left[\frac{a_0}{2}\left(3n^2- l(l+1)\right)\right]
\right] \\
  &= \frac{n^2a_0^3}{4}\left[\frac{1}{2}\left(35n^4-21n^2l(l+1)+7n^2 \right)
    -\frac{3}{8}\left[12n^2l(l+1)-4l^2(l+1)^2 -24n^2+8l(l+1)\right]\right]\\
  &= \frac{n^2a_0^3}{8}\left[\left(35n^4-21n^2l(l+1)+7n^2 \right)-
  \left(9n^2l(l+1)-3l^2(l+1)-18n^2+6l(l+1) \right)\right]\\
\Aboxed{\big<r^3\big> &= \frac{n^2a_0^3}{8}\left[35n^4 - 30n^2l(l+1) +25n^2+3l^2(l+1)-6l(l+1)
\right]}
\end{align*}
\textbf{Part b}\\
\\
\textbf{Plugin $s = -1$}\\
\\
with $s=-1$, Kramers' relation becomes:
\[
  a_0 \big<r^{-2}\big>-\frac{1}{4}\left[4l(l+1)\right]a_0^2\big<r^{-3}\big>=0 
\]
solve for $\big<r^{-3}\big>$ 
\begin{align*}
  \Aboxed{\big<r^{-3}\big> &= \frac{1}{l(l+1)a_0}\bigg<\frac{1}{r^2}\bigg>}
\end{align*}
\textbf{Part c}\\
\\
Using the result for $\big<r^{-2}\big>$ obtained in Problem 6.28, we get 
\[\bigg<\frac{1}{r^2}\bigg> = \frac{1}{(l+1/2)n^3a_0^2}\]
then
\[
  \big<r^{-3}\big> = \frac{1}{l(l+1)a_0}\left(\frac{1}{(l+1/2)n^3a_0^2} \right)
\]
\[
  \boxed{\big<r^{-3}\big> = \frac{1}{l(l+1/2)(l+1)n^3a_0^3}}
\]
Finally, we can see that this result agrees with that stated in Equation 6.63.
%%%%%%%%%%%%%%%%%%%%%%%%%%%%%%%%%%%%%%%%%%%%%%%%%%%%%%%%%%%%%%%%%%%%%%%%%%%%%%%%%%%%%%%%%%%%%%%%%%
%%%%%%%%%%%%%%%%%%%%%%%%%%%%%%%%%%%%%%%%%%%%%%%%%%%%%%%%%%%%%%%%%%%%%%%%%%%%%%%%%%%%%%%%%%%%%%%%%%
%%%%%%%%%%%%%%%%%%%%%%%%%%%%%%%%%%%%%%%%%%%%%%%%%%%%%%%%%%%%%%%%%%%%%%%%%%%%%%%%%%%%%%%%%%%%%%%%%%
\section*{Problem 6.33}
Calculate the wavelength, in centimeters, of the photon emitted under a hyperfine 
transition in the ground state $(n=1)$ of deuterium. Deuterium is "heavy" hydrogen,
with an extra neutron in the nucleus. The proton and neutron bind together to form a 
deuteron, with spin 1 and magnetic moment
\[\vec{\mu}_d = \frac{g_d e}{2m_d}\vec{S}_d\]
the deuteron g-factor is 1.71.
\subsection*{Solution}
We know that 
\[
  \vec{\mu}_d = \frac{g_d e}{2m_d}\vec{S}_d\quad \text{and}\quad g_d = 1.71 
\]
The hyperfine splitting is given by:
\[
  E_{hf}^1 = \frac{\mu_0 g_p e^2}{3\pi m_p m_e a_0^3}\big<\vec{S}_p\cdot\vec{S}_e\big>
\]
, but in this case we are interested in deuterium, so we have that:
\[
  E_{hf}^1 = \frac{\mu_0 g_d e^2}{3\pi m_d m_e a_0^3}\big<\vec{S}_d\cdot\vec{S}_e\big>
\]
We also know that
\[
  \big<\vec{S}_p\cdot\vec{S}_e\big> =\frac{1}{2}\left( S_{tot}^2 - S_e^2 -S_d^2\right), 
  \qquad S_n^2 = \hbar^2s_n(s_n+1)
\]
here 
\[
  S_d = 1  \Rightarrow S_d^2 = \hbar^2 1(1+1) =2\hbar^2
\]
\[
  S_e = \frac{1}{2} \Rightarrow S_e^2 = \hbar^2 \frac{1}{2}\left(\frac{1}{2}+1\right) =
  \frac{3}{4}\hbar^2
\]
for the total spin we will have to cases
\[
  S_{tot} =
  \begin{cases}
    \frac{3}{2},\quad \text{when } s = \frac{1}{2}\\
    \frac{3}{4}, \quad \text{when } s = -\frac{1}{2}
  \end{cases}
\]
then 
\[
  S^2_{tot} =
  \begin{cases}
    \frac{15}{4}\hbar^2,\quad \text{when } s = \frac{1}{2}\\
    \frac{3}{4}\hbar^2, \quad \text{when } s = -\frac{1}{2}
  \end{cases}
\]
\[
\big<\vec{S}_p\cdot\vec{S}_e\big> = \begin{cases}
  \frac{1}{2}\left(\frac{15}{4}\hbar^2 - \frac{3}{4}\hbar^2 -2\hbar^2\right)&= \frac{1}{2}\hbar^2\\
\frac{1}{2}\left(\frac{3}{4}\hbar^2 - \frac{3}{4}\hbar^2 -2\hbar^2\right)&= -\hbar^2\\
\end{cases}
\]
Plugin this into the hyperfine splitting, we get
\[
 E_{hf}^1 =  \frac{\mu_0 g_d e^2}{3\pi m_d m_e a_0^3} \begin{cases}
    \frac{1}{2}\hbar^2\\
    -\hbar^2
  \end{cases}
\]
Now, let's compute $\Delta E$
\[
  \Delta E = \frac{\mu_0 g_d e^2\hbar^2}{2\pi m_d m_e a_0^3}
\]
but 
\[
  \mu_0 \epsilon_0 = \frac{1}{c^2} \Rightarrow \mu_0 =\frac{1}{\epsilon_0c^2}
\]
So, 
\[
 \Delta E = \frac{ g_d e^2\hbar^2}{2\pi\epsilon_0 c^2 m_d m_e a_0^3}
\]
Rewrite this using Bohr radius
\[
  a_0 = \frac{4\pi \epsilon_0 \hbar^2}{me^2}
\]
\[
\Rightarrow \Delta E = \frac{ 2g_d\hbar^4}{ m_d m_e^2 c^2 a_0^4} 
\]
Now, using the energy gap of the hydrogen 
\[
  \Delta E_{h} = \frac{4g_p\hbar^4}{3m_pm_e^2c^2a_0^4}\quad
  \text{(Griffiths Equation 6.92)}
\]
we can rewrite as follows
\[
  \Delta E = \frac{3g_dm_p}{2g_pm_d}\Delta E_{h}
\]
The energy of a photon is given by 
\[
  E = h\nu = h \frac{c}{\lambda} \Rightarrow \lambda = \frac{h c}{E}
\]
then
\[
  \lambda_d = \frac{2g_pm_d}{3g_dm_p}\frac{hc}{\Delta E_{h}}
\]
but
\[
  \Delta E_{h} = \frac{hc}{\lambda_{h}} \Rightarrow  \Delta E_{h}
  \lambda_{h} = hc
\]
then, 
\[
  \lambda_d = \frac{2g_pm_d}{3g_dm_p}\lambda_h 
\]
pluggin $g_p=5.59, g_d = 1.71,m_d/m_p = 2, \lambda_h = 21\,cm $
\[
  \lambda_d = \frac{2(5.59)(2)}{3(1.71)}(21\,cm)
\]
\[
 \boxed{\lambda_d = 91.53\,cm}
\]












\end{document}

