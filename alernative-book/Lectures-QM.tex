\documentclass[12pt]{article}
\usepackage{/home/pacman/Documents/tex-packages/NotesTeXV3}
%\usepackage{marginnote, sidenotes, fancyhdr, titlesec, geometry, tcolorbox}
%/Path/to/package should be replaced with package location
\usepackage{bbold}
\usepackage{cancel}
\hypersetup{
    colorlinks,
    citecolor=black,
    filecolor=black,
    linkcolor=red,
    urlcolor=red
}


\title{{\Huge Lectures on Quantum Mechanics}\\{\Large{SEM02-2023}}}
\author{Gabriel Balarezo\footnote{\href{https://google.com/}{\textit{https://github.com/GabrielBJ}}}}

\affiliation{Yachay Tech University \\ School of Nanotechnology and Physical Sciences \\ Urcuqui, Ecuador}
\emailAdd{balarezog961@gmail.com}
\begin{document}
  \maketitle
  \flushbottom
  \newpage
  \pagestyle{fancynotes}
  \part*{Introduction}
Quantum mechanics is certainly the most successful theory of the physical world at small scales. It is also one of the most mysterious.
  
  Some notes here.\sn{With some additional sidenotes}
\newpage
  \part{Time Independent Non-degenerate Perturbation Theory}

  \section{Lecture 1}
%%%%%%%%%%%%%%%%%%%%%%%%%%%%%%%%%%%%%%%%%%%%%%%%%%%%%%%%%%%%%%%%%%%%%%%%%%%%%%%%%%%%%%%%%%%%%%%%%%%
%%%%%%%%%%%%%%%%%%%%%%%%%%%%%%%%%%%%%%%%%%%%%%%%%%%%%%%%%%%%%%%%%%%%%%%%%%%%%%%%%%%%%%%%%%%%%%%%%%%
%%%%%%%%%%%%%%%%%%%%%%%%%%%%%%%%%%%%%%%%%%%%%%%%%%%%%%%%%%%%%%%%%%%%%%%%%%%%%%%%%%%%%%%%%%%%%%%%%%%
Suppose we have a "known" system with Hamiltonian, eigenfunctions and eigenenergies $\displaystyle{H_0}$,  $\displaystyle{\Psi_n}$ and $\displaystyle{E_n}$ respectively.
  \begin{equation*}
    \hat{H} |\Psi_n \rangle = E_n |\Psi_n \rangle 
\end{equation*}
where
\sn{The student may remember the Kronecker delta from his/her Quantum Mechanics I course.}
$$\langle \hat{\Psi} |\hat{H}|\hat{\Psi} \rangle = E_n,\,\,\,\,\,\,\,\,
	\langle \Psi_n | \Psi_{n'} \rangle = \delta_{nn'} $$
So that $\{\displaystyle{\Psi_n}\}$ forms a orthonormal basis over the Hilbert space of $\displaystyle{H_0}$. 
We apply a perturbative potential $\tilde{V}$ \sn{$\tilde{V}$ is very small compared to $H_0$.} to $H_0$, yielding a new Hamiltonian. 
for the perturbated system,
\begin{equation}
\label{eq:1}
  \tilde{H} = \hat{H_0} + \lambda \tilde{V},\,\,\,\,\text{let}\,\,\lambda \in [0,1]
\end{equation}
a "bookkeeping parameter" to keep track of the relative signs of from in order of the perturbated 
potential.
\subsection{Method}
Express the wave functions and eigenenergies of the perturbated Hamiltonian $H$, $\tilde{\psi}$, $\tilde{E}_n$ 
as a power series equation in orders of $\lambda$.
\begin{align*}
	\tilde{E}_n & = \tilde{E}^{(0)}_0 + \lambda \tilde{E}^{(1)}_n + \lambda^2 \tilde{E}^{(2)}_n
	+ \cdots = \sum_{k} \lambda^{k} \tilde{E}^{(k)}_n\\
	\tilde{\psi}_n & = \tilde{\psi}^{(0)}_n + \lambda \tilde{\psi}^{(1)}_n + \lambda^2 \tilde{\psi}^{(2)}_n 
	+ \cdots = \sum_{k} \lambda^k \tilde{\psi}^{(k)}_n
\end{align*}
Now we write the Time independent Schr\"odinger equation for the perturbated Hamiltonian,
\begin{align}
  \tilde{H} \big|\tilde{\psi}_n \big > & = \tilde{E}_n \big|\tilde{\psi}_n \big >\\
	(\hat{H} + \hat{V}) \sum_{k} \lambda^k \big| \psi^{(k)}_n \big > & = \sum_{k} \lambda^k \tilde{E}^{(k)}_n \sum_{k}
	\lambda^k \big|\psi^{(k)}_n \big>\\
	\sum_{k}(\lambda^k \hat{H}_0 + \lambda^{k+1} \hat{V}) \big|\hat{\psi}^{(k)}_n \big > & = \sum_{k'} \sum_{k}
	\lambda^{k'+k} \hat{E}^{(k)}_n \big| \hat{\psi}^{(k)}_n \big >
\end{align}
\begin{equation}
	\sum_{k} \left(\lambda^k \hat{H}_0 + \lambda^{k+1} \hat{V} - \sum_{k'} \lambda^{k'+ k}  
	\hat{E}^{(k')}_n \right ) \big|\psi^{(k)}_n \big > = 0 
\end{equation}
\begin{equation}
  \label{eq:6}
	\sum_{k} \lambda^{k} \left(\hat{H}_0 \big| \hat{\psi}^{(k)}_n \big > + \underbrace{(1 - \delta_
	{k0})}_{k\neq 0} \hat{V} \big| \hat{\psi}^{(k-1)} \big > - \sum_{k=0}^{k} \hat{E}^{(k')}_n 
	\big | \hat{\psi}^{(k-k')}_n \big > \right ) = 0
\end{equation}
Since \ref{eq:6} is true for any value of $\lambda$, it must hold for each power of $\lambda$ separately:
\begin{align*}
	\mathcal{O}(1)_{k=0} & = \hat{H} \big | \hat{\psi}^{(0)}_n \big > + (1 - \delta_{k0}) \hat{V}
	\big | \hat{\psi}^{(-1)}_n \big >\\
	& = \sum_{k' = 0}^{0} \hat{E}^{(k')}_n \big| \hat{\psi}^{(0-k')}_n \big>\\
	\Longrightarrow \hat{H}_0 \big| \hat{\psi}^{(0)}_n \big > & = \hat{E}^{(0)}_n \big| \hat{\psi}^{(0)}_n \big>\\
  \big < \psi_n \big | \underbrace{H_0}_{\mathclap{\text{hermitian}}} \big| \hat{\psi}^{(0)}_n \big > & = \hat{E}^{(0)}_n \big < \psi_n   \big |
	\hat{\psi}^{(0)}_n\big>
\end{align*}
\begin{equation*}
  \big < \psi_{n'} \big|\hat{H}_0 ^\dag = E_{n'} \big < \psi_{n'} \big |
\end{equation*}
\begin{equation*}
	(E_{n'} - \hat{E}^{(0)}_n)\big < \psi_{n'} \big|\hat{\psi}^{(0)}_n \big > = 0
\end{equation*}
So,
\begin{equation}
  \hat{E}^{(0)}_n = E_n\,\,\,\text{or}\,\,\,\big < \psi_n \big| \hat{\psi}^{(0)}_n \big > = 0 
\end{equation}
\noindent
If the system $\hat{H}_0$ is nondegenerate: 
\[ \Longrightarrow \big|\hat{\psi}^{(0)}_n\big> = \big| \psi_n \big >\]
If the system is degenerate, then:
\begin{equation*}
	\hat{H}_0 \big|\psi_{n'} \big > = E_n \big| \psi_{n'} \big >\,\,\,\,\text{for } n' \in \{n, \cdots, n+k\}
\end{equation*}
Then:
\begin{equation*}
	\big | \hat{\psi}^{(0)}_n \big > = \sum_{k'=0}^{k} C_{n+k'} \big|\psi_{n+k'} \big >
\end{equation*}
\begin{equation*}
	\hat{E}_n = E_n \mathcal{O}(\lambda)
\end{equation*}
\begin{equation*}
	\big | \hat{\psi}_n \big > = | \psi_n\big > + \mathcal{O}(\lambda)
\end{equation*}
\begin{equation*}
\mathcal{O}(\lambda): \hat{H}_0 \big|\hat{\psi}^{(1)}_n \big> + \hat{V} \big|\hat{\psi}^{(0)}_n \big>
= E_n^{(0)} \big|\hat{\psi}^{(1)}_n \big> + \hat{E}^{(1)}_n  \big|\hat{\psi}^{(0)}_n \big> 
\end{equation*}
multiply by $\big<\psi_{n'} \big|$:
\begin{equation*}
\big<\psi_{n'} \big|\hat{H}_0 \big|\hat{\psi}^{(1)}_n \big> + \big<\psi_{n'} \big|\hat{V} \big|\hat{\psi}^{(0)}_n \big>
= E_n^{(0)}\big<\psi_{n'}\big|\hat{\psi}^{(1)}_n \big> + \hat{E}^{(1)}_n \big<\psi_{n'} \big|\hat{\psi}^{(0)}_n \big> 
\end{equation*}
\begin{equation*}
	\big<\psi_{n'} \big|\hat{H}_0 = E_{n'} \big<\psi_{n'} \big|
\end{equation*}
$$
\big|\hat{\psi}^{(0)}_n \big> = \big|\psi_n \big>\,\,\,\,\text{(non-degenerate)},\,\,\,\hat{E}^{(0)}_n = E_n
$$
\begin{equation}
	(E_{n'}-E_n)\big< \psi_{n'} \big|\hat{\psi}^{(1)}_n \big> + \big < \psi_{n'} \big| \hat{V}\big|\psi_n \big>
	= \hat{E}^{(1)}_n \delta_{n'n}
\end{equation}
\textbf{if} $\mathbf{n'= n}$:
\begin{equation*}
	\cancelto{0}{(E_{n'}-E_n)} \big< \psi_{n'} \big|\hat{\psi}^{(1)}_n \big> + \big < \psi_{n'} \big| \hat{V}\big|\psi_n \big>
	= \hat{E}^{(1)}_n \delta_{n'n}
\end{equation*}
\begin{align}
	\Longrightarrow \hat{E}^{(1)}_n & = \big < \psi_{n} \big| \hat{V}\big|\hat{\psi}^{(1)}_n \big>\nonumber\\
	\hat{E}_n & = E_n + \big < \psi_{n'} \big| \hat{V}\big|\hat{\psi}^{(1)}_n \big> + \mathcal{O}(\lambda^2)\nonumber\\
	\big | \hat{\psi}_n \big > & = \big| \psi_n \big> + \mathcal{O} (\lambda)
\end{align}
\textbf{if} $\mathbf{n'\neq n}$:
\begin{equation*}
	(E_{n'}-E_n)\big< \psi_{n'} \big|\hat{\psi}^{(1)}_n \big> = -\big < \psi_{n'} \big| \hat{V}\big|\psi_n \big>
	+ \hat{E}^{(1)}_n \cancelto{0}{\delta_{n'n}}
\end{equation*}
\begin{equation}
	\big< \psi_{n'} \big|\hat{\psi}^{(1)}_n \big> = \frac{\big < \psi_{n'} \big| \hat{V}\big|\psi_n \big>}{E_n - E_{n'}}
\end{equation}
Since $\{ \psi_n\}$ forms an orthonormal basis,
\begin{equation*}
	\Longrightarrow \sum_{n'} \big|\psi_{n'} \big> \big< \psi_{n'} \big | = \mathbb{1}\,\,\,\text{Identity matrix}
\end{equation*}
\begin{align}
	\big| \hat{\psi}^{(1)}_n \big > = \mathbb{1} \big|\hat{\psi}^{(1)}_n\big> & = \sum_{n'} \big|\psi_{n'}
	\big> \big < \psi_{n'} | \hat{\psi}^{(1)}_n \big>\nonumber\\
	& = \sum_{n'=n} \frac{\big < \psi_{n'} \big| \hat{V}\big|\psi_n \big>}{E_n- E_{n'}} \big|\psi_{n'} \big> 
\end{align}
Since $\big |\hat{\psi}^{(0)}_n \big> = \big| \psi_{n} \big>$, then $\big |\hat{\psi}^{(1)}_n \big>$
cannot depend on $\big|\psi_n\big>$. 
\newpage
%%%%%%%%%%%%%%%%%%%%%%%%%%%%%%%%%%%%%%%%%%%%%%%%%%%%%%%%%%%%%%%%%%%%%%%%%%%%%%%%%%%%%%%%%%%%%%%%%%%
%%%%%%%%%%%%%%%%%%%%%%%%%%%%%%%%%%%%%%%%%%%%%%%%%%%%%%%%%%%%%%%%%%%%%%%%%%%%%%%%%%%%%%%%%%%%%%%%%%%
%%%%%%%%%%%%%%%%%%%%%%%%%%%%%%%%%%%%%%%%%%%%%%%%%%%%%%%%%%%%%%%%%%%%%%%%%%%%%%%%%%%%%%%%%%%%%%%%%%%
\section{Lecture 2}
In the previous lecture we defined the perturbed hamiltonian
\sn{$\tilde{H}$: perturbed hamiltonian, $\hat{H}_0$: unperturbed hamiltonian, $\lambda$: bookkeeping parameter,\\ $\hat{V}$: perturbing potential.}
\ref{eq:1}  as:
\begin{equation*}
  \tilde{H} = \hat{H}_0 + \lambda \hat{V}
\end{equation*}
\noindent
$\hat{V}$ needs to be smaller than $\hat{H}_0$, $\lambda$ keeps track of this, i.e., dependence on $\hat{V}$.
\noindent
\\
Writing the Schr\"odinger equation:
\begin{equation*}
	(\hat{H}_o + \lambda \hat{V}) \hat{\psi}_n = \tilde{E}_n \tilde{\psi}_n
\end{equation*}
$\tilde{E}_n$ and $\tilde{\psi}_n$ are eigenenergies and wavefunctions for the perturbated hamiltonian.
\\
Expand $\tilde{E}_n$ and $\tilde{\psi}_n$ in power of $\lambda$:
\begin{align*}
	\tilde{E}_n & = \hat{E}^{(0)}_n + \lambda \tilde{E}^{(1)}_n + \lambda^2 \tilde{E}^{(2)}_n + \mathcal{O}(\lambda^3)\\
	\tilde{\psi}_n & = \hat{\psi}^{(0)}_n + \lambda \tilde{\psi}^{(1)}_n + \lambda^2 \tilde{\psi}^{(2)}_n + \mathcal{O}(\lambda^3)
\end{align*}
Substitute for $\tilde{E}_0$ and $\tilde{\psi}_n$ into the (SE) and collect terms of like powers of $\lambda$.
\begin{equation*}
	\mathcal{O}(1): \hat{H}_0 \big|\tilde{\psi}^{(0)}_n \big> = \tilde{E}^{(0)}_0 \big| \psi_n^{(0)} \big>
\end{equation*}
multiply by $\big< \psi_{n'}\big|$:
\begin{equation*}
	\big< \psi_{n'}\big|\hat{H}_0 \big|\tilde{\psi}^{(0)}_n \big> = \tilde{E}^{(0)}_0 \big< \psi_{n'}\big| \psi_n^{(0)} \big>
\end{equation*}
$\hat{H}_0 = H_0^\dag$ since $\hat{H}_0$ is hermitian.
\begin{equation*}
	\big < \psi_{n'} \big| \hat{H}^\dag_0 = E_{n'} \big< \psi_{n'} \big|\,\,\,\text{(SE) for } \hat{H}_0
\end{equation*}
\begin{equation*}
	\Longrightarrow E_{n'} \big< \psi_{n'} \big| \psi_n^{(0)} \big> = 
	\hat{E}^{(0)}_0\big< \psi_{n'} \big| \psi_n^{(0)} \big> 
\end{equation*}
$$E^{(0)}_{n'} = E_{n'}\,\,\,\,\text{or}\,\,\,\,\big< \psi_{n'} \big| \psi_n^{(0)} \big> = 0$$
$\mathbb{1} = \sum_{n'} \big|\psi_{n'}\big> \big< \psi_{n'} \big|$
\\
\\
$\tilde{\psi}_n^{(0)} = \sum_{n'} \big< \psi_{n'} \big| \psi_n^{(1)} \big> $
\\
\\
If $E_n$ is k-level degenerate, then:
\begin{equation}
	\big| \tilde{\psi}^{(0)}_n \big> = \sum_{n'=n}^{n+k-1} \big< \psi_{n'} \big| \psi_n^{(0)} \big>
	\big|\psi_{n'} \big>
\end{equation}
Consider $\hat{H}_0$ to be nondegenerate: 
$$
	\Longrightarrow E^{(0)}_n = E_n,\,\,\,\,\big|\big< \psi_n | \tilde{\psi}^{(0)}_n \big> \big| ^2 = 0
$$
$$\sum_{n'=n}^{n+k-1}\big| \big< \psi_{n'} \big| \psi_n^{(0)} \big> \big|^2 = 1$$

Since $ \big< \psi_{n'} \big| \psi_n^{(0)} \big> = 0\,\,\text{for }n'\neq n$:
$$\Longrightarrow \big| \tilde{\psi}^{(0)}_n \big > = \big| \psi_{n'} \big>$$
\\
The $n^{th}$ eigenenergy of the perturbated system (hamiltonian) $\tilde{E}_n$ is the $n^{th}$ 
eigenenergy of the unperturbed system to Zeroth order in the perturbation, i.e $\mathcal{O}(\lambda^0)
= \mathcal{O}(1),\,\,\lambda = 0 $.
\\
$\Longleftrightarrow$ If we turn the perturbation off we get the eigenenergy of the unperturbed system.
\\
$\because$ The $n^{th}$ wavefunction for the perturbed hamiltonian $\tilde{\Psi}_n$ is a linear 
combination of the unperturbed hamiltonian's degenerate wavefunctions with energy $E_n$ to Zeroth
order in the perturbation, i.e $\mathcal{O}(\lambda^0) = \mathcal{O}(1)\,\,\,\text{or}\,\,\,\lambda = 0$ 
\begin{equation*}
	\Longrightarrow \big|\hat{\psi}^{(0)}_n \big> =\sum_{n'=n}^{n+k-1}
	\big< \psi_{n'} \big|\tilde{ \psi}_n^{(0)} \big> \big|\psi_{n'}\big>
\end{equation*}
,where
$$\sum_{n'=n}^{n+k-1}\big| \big< \psi_{n'} \big| \tilde{\psi}_n^{(0)} \big> \big|^2 = 1$$
Even if we "turn off" the perturbation, it can still left the degeneracy of $\{\big|\psi_n\big>,
\cdots, \big|\psi_{n+k-1}\big>\}$ for k-fold degenerate systems, unless the unperturbed wavefunctions
are eigenstates of the perturbation.\\
\\
Now let's assume $\hat{H}_0$ is non-degenerate
$$\Longrightarrow \big| \big< \psi_{n'} \big| \tilde{\psi}_n^{(0)} \big> \big|^2 = 1$$
wlog $\psi_n = \tilde{\psi}^{(0)}_n$ (Choice of 1 for phase factor).
\begin{align*}
	\tilde{E}_n & = E_n + \mathcal{O}(\lambda)\\
	\tilde{\psi}_n & = \psi_n + \mathcal{O}(\lambda)
\end{align*}
Gather together terms of order 1 in $\lambda$ in the (SE):
\begin{equation}
	\mathcal{O}(\lambda):\lambda \left( \hat{H}_o \big|\tilde{\psi}^{(1)}_n \big> + 
	\hat{V}\big| \tilde{\psi}^{(0)}_n \right) = 
	\lambda \left(E_n \big|\tilde{\psi}^{(1)}_n \big> + E^{(1)}_n \big|\psi_n \big> \right)
\end{equation}
Set $\lambda = 1$
\sn{This is a recursive equation relating the 1st order corrections in $\tilde{E}_n$ and $\tilde{\psi}_n$
to the Zeroth order corrections, in the unperturbed eigenenergy and wavefunction, $E_n$ and $\psi_n$
respectively $\because$ the $\mathcal{O}^{th}$ relates the kth order corrections to all previous orders.
}, 
\begin{equation}
	\hat{H}_o \big|\tilde{\psi}^{(1)}_n \big> + 
	\hat{V}\big| \tilde{\psi}^{(0)}_n \big> = 
	E_n \big|\tilde{\psi}^{(1)}_n \big> + E^{(1)}_n \big|\psi_n \big> 
\end{equation}
\subsection{Two Unknowns}
Use the fact that:
\[\big< \psi_{n'} \big| \psi_n\big> = \delta_{n'n}\,\,\,\,\text{and}\,\,\,\,\hat{H}_o \big|\psi_{n'}
\big> = E_{n'} \big | \psi_{n'} \big>\]
To obtain equations for $\tilde{E}^{(1)}_n$ and $\tilde{\psi}^{(1)}_n$.\\
\\
multiply by $\big< \psi_{n'} \big|$: 
\begin{equation*}
	\big< \psi_{n'} \big|\hat{H}^\dag_o \big|\tilde{\psi}^{(1)}_n \big> + 
	\big< \psi_{n'} \big|\hat{V}\big| \tilde{\psi}^{(0)}_n \big> = 
	E_n \big< \psi_{n'} \big|\tilde{\psi}^{(1)}_n \big> + E^{(1)}_n \underbrace{\big< \psi_{n'}
	\big|\psi_n \big>}_{\delta_{n'n}} 
\end{equation*}
\begin{equation}
\label{eq:2.4}
	\left(E_{n'} - E_n\right)\big< \psi_{n'} \big|\tilde{\psi}^{(1)}_n \big> +
	\big< \psi_{n'} \big|\hat{V}\big| \psi_n \big> = \tilde{E}^{(1)}_n \delta_{n'n}
\end{equation}
$\big< \psi_{n'} \big|\tilde{\psi}^{(1)}_n \big>  $: Coefficient of $\big|\psi_{n'} \big>$.\\
\\
$\tilde{E}^{(1)}_n $: First order correction to the eigenenergy.
\\
\\
If $n'=n$, \ref{eq:2.4} yields to: $\tilde{E}^{{(1)}}_n$ 
\begin{equation*}
	\boxed{\tilde{E}^{(1)}_n =	\big< \psi_{n} \big|\hat{V}\big| \psi_n \big>  }
\end{equation*}
is the expectation value of the perturbation in the $n^{th}$ eigenstate of the unperturbed hamiltonian.
\\
\\
If $n'\neq n$, \ref{eq:2.4} yields to: coefficients of $\big | \tilde{\psi}^{(1)}_n\big>$ and $\lambda = 1$.
\begin{equation*}
	\big < \psi_{n'} |\hat{\psi}^{(1)}_n \big> = \frac{\big< \psi_{n'}\big|\hat{V}\big|\psi_n
	\big>}{E_n - E_{n'}}
\end{equation*}
\begin{equation*}
	\big|\tilde{\psi}^{(1)}_n \big> = \sum_{n' \neq n}
	\frac{\big< \psi_{n'}\big|\hat{V}\big|\psi_n\big>}{E_n - E_{n'}}\big|\psi_{n'}\big>
\end{equation*}
\begin{equation*}
	\tilde{E}_n = E_n +\big< \psi_{n}\big|\hat{V}\big|\psi_n\big> + \mathcal{O}(\lambda^2) 
\end{equation*}
\begin{equation}
	\big| \tilde{\psi}_n \big> = \big| \psi_n \big> +\sum_{n' \neq n}
	\frac{\big< \psi_{n'}\big|\hat{V}\big|\psi_n\big>}{E_n - E_{n'}}\big|\psi_{n'}\big> + \mathcal{O}(\lambda^2)
\end{equation}
The first order correction to the eigenenergy is the energy of the perturbation of you stay in the
$n^{th}$ eigenstate of the unperturbed system. So $V$ \sn{\textbf{Note:} if $\hat{V}$ (the perturbation) is an odd function:
\begin{align*}
  \hat{E}^{(1)}_n &= \int d^3 \vec{r}\big|\underbrace{\psi(\vec{r})}_\text{even}
	\big|^2 \overbrace{\hat{V}(\vec{r}, t)}^\text{odd}\\
  &= 0
\end{align*}} has to be "weak" so that you can remain in
$\big|\psi_n\big>$ under its action.\\
\\
The first order corrections to wavefunctions are proportional to the matrix elements: 
$$\big<\psi_{n'}\big|\hat{V}\big|\psi_n\big>,$$
i.e., that under the action of the perturbation you can transition from $n^{th}$ to $n^{'th}$ 
eigenstate of the unperturbed hamiltonian and inversely proportional to the difference (energy)
between the eigenenergies of the $n^{th}$ and $n^{'th}$  eigenstates of the perturbed hamiltonian.\\
\\
\textbf{Note:}
\begin{equation*}
	\big|\big<\psi_{n'} \big| \hat{V}\big| \psi_n \big> \big| \ll \big|E_n - E_{n'}\big|
\end{equation*}
Otherwise our expression for the wavefunctions of the perturbed system won't converge.
$$	\big|\big<\psi_{n'} \big| \hat{V}\big| \psi_n \big> \big| \ll \big|E_n - E_{n'}\big|$$ 
for our application of perturbation theory to apply.
$$\Longrightarrow \lambda \gg 1$$
\begin{equation*}
	\mathcal{O}(\lambda ^2): \hat{H}_o \big|\tilde{\psi}^{(2)}_n\big> + \hat{V}\big|\tilde{\psi}^{(1)}_n\big>
	= E_n \big|\tilde{\psi}^{(2)}_n \big> + \tilde{E}^{(1)}_n \big|\tilde{\psi}^{(1)}_n \big>
	+ \tilde{E}^{(2)}_n \big| \psi_n \big>
\end{equation*}
Multiply by $\big< \psi_{n'} \big|$:
\begin{equation*}
	\big< \psi_{n'} \big|\hat{H}_o \big|\tilde{\psi}^{(2)}_n\big> + 
	\big< \psi_{n'} \big|\hat{V}\big|\tilde{\psi}^{(1)}_n\big>
	= E_n \big< \psi_{n'} \big|\tilde{\psi}^{(2)}_n \big> + 
	\tilde{E}^{(1)}_n \big< \psi_{n'} \big|\tilde{\psi}^{(1)}_n \big>
	+ \tilde{E}^{(2)}_n \big< \psi_{n'} \big| \psi_n \big>	
\end{equation*}
\begin{multline*}
	(E_{n'} - E_n)	\big< \psi_{n'} \big| \hat{\psi}^{(2)}_n \big> + \big<\psi_{n'} \big|
	\hat{V} \sum_{k\neq n} \frac{\big< \psi_k \big| \hat{V} \big| \psi_n \big>}{E_n - E_k}
	\big|\psi_k \big>\\
	= \big<\psi_n \big| \hat{V}\big| \psi_n\big>\big<\psi_{n'}\big|\sum_{n=k}\frac{\big<\psi_k\big|
	\hat{V}\big|\psi_n\big>}{E_n - E_k}\big|\psi_k\big> + \tilde{E}^{(2)}_n \delta_{n'n} 
\end{multline*}
\begin{multline*}
	(E_{n'} - E_n)	\big< \psi_{n'} \big| \hat{\psi}^{(2)}_n \big> +
	\sum_{k=n} \frac{\big<\psi_k\big|\hat{V}\big|\psi_n\big>\big<\psi_{n'}\big|\hat{V}\big|\psi_k\big>}
	{E_n - E_k}\\
	= \frac{\big<\psi_n\big|\hat{V}\big|\psi_n\big>\big<\psi_{n'}\big|\hat{V}\big|\psi_n\big>}
	{E_n - E_{n'}}\bigg(1-\delta_{n'n}\bigg) + \tilde{E}^{(2)}_n \delta_{n'n}
\end{multline*}
If $n' = n$:
\begin{equation}
	\boxed{\tilde{E}^{(2)}_n = \sum_{k\neq n} \frac{\big|\big<\psi_k\big|\hat{V}\big|\psi_n\big>\big|^2}
	{E_n - E_k}}
\end{equation}
If $n' \neq n$:
\begin{equation}
	\big<\psi_{n'}\big|\tilde{\psi}^{(2)}_n \big> = \sum_{k\neq n} \frac{\big<\psi_k\big|\hat{V}\big|
	\psi_n\big>\big<\psi_{n'}\big|\hat{V}\big|\psi_{k}\big>}{(E_n - E_k)(E_n - E_{n'})} + 
	\frac{\big<\psi_n\big|\hat{V}\big|\psi_n\big>\big<\psi_{n'}\big|\hat{V}\big|\psi_n\big>}
	{E_n - E_{n'}}
\end{equation}
\begin{equation}
	\big|\tilde{\psi}^{(2)}_n \big> = \sum_{n'\neq n} \bigg| \psi_{n'}\bigg> \left[\frac{\big<\psi_{n'}\big|
	\hat{V}\big|\psi_k\big>\big<\psi_k\big|\hat{V}\big|\psi_n\big>}{(E_n - E_k)(E_n-E_{n'})}
	+\frac{\big<\psi_{n'}\big|\hat{V}\big|\psi_n\big>\big<\psi_n|\hat{V}\big|\psi_n\big>}
	{E_n - E_{n'}}\right] 
\end{equation}
\newpage
%%%%%%%%%%%%%%%%%%%%%%%%%%%%%%%%%%%%%%%%%%%%%%%%%%%%%%%%%%%%%%%%%%%%%%%%%%%%%%%%%%%%%%%%%%%%%%%%%%%
%%%%%%%%%%%%%%%%%%%%%%%%%%%%%%%%%%%%%%%%%%%%%%%%%%%%%%%%%%%%%%%%%%%%%%%%%%%%%%%%%%%%%%%%%%%%%%%%%%%
%%%%%%%%%%%%%%%%%%%%%%%%%%%%%%%%%%%%%%%%%%%%%%%%%%%%%%%%%%%%%%%%%%%%%%%%%%%%%%%%%%%%%%%%%%%%%%%%%%%
%%%%%%%%%%%%%%%%%%%%%%%%%%%%%%%%%%%%%%%%%%%%%%%%%%%%%%%%%%%%%%%%%%%%%%%%%%%%%%%%%%%%%%%%%%%%%%%%%%%
\section{Lecture 3}























\end{document}
